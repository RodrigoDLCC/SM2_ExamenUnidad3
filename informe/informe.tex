\documentclass[12pt, a4paper]{article}
\usepackage[spanish]{babel}
\usepackage[utf8]{inputenc}
\usepackage{graphicx}
\usepackage{hyperref}
\usepackage{booktabs}
\usepackage{tabularx}
\usepackage{geometry}
\usepackage{enumitem}

% Configuración de página
\geometry{margin=2.5cm}
\setlength{\parindent}{0pt}
\setlength{\parskip}{1em}

% Estilo de tablas
\newcolumntype{Y}{>{\raggedright\arraybackslash}X}

\begin{document}

\begin{center}
\Large\textbf{Informe del Examen Unidad III} \\
\large\textbf{Automatización de calidad con GitHub Actions}
\end{center}

\section*{Datos del Estudiante}
\begin{tabular}{ll}
\textbf{Curso:} & Desarrollo de Aplicaciones Móviles \\
\textbf{Fecha:} & \today \\
\textbf{Nombre:} & [Nombre Completo del Estudiante] \\
\textbf{Repositorio:} & \url{https://github.com/[usuario]/SM2_ExamenUnidad3} \\
\end{tabular}

\section*{Evidencias del Proyecto}

\subsection*{1. Estructura de Carpetas}
\begin{itemize}
\item Se implementó la estructura requerida en el repositorio:
\begin{itemize}
\item \texttt{.github/workflows/} $\rightarrow$ Contiene \texttt{quality-check.yml}
\item \texttt{test/} $\rightarrow$ Contiene \texttt{main\_test.dart} con pruebas unitarias
\end{itemize}
\end{itemize}

\begin{center}
\includegraphics[width=0.8\textwidth]{folder_structure.png}
\captionof{figure}{Estructura de directorios del proyecto}
\end{center}

\subsection*{2. Workflow de GitHub Actions}
\begin{verbatim}
name: Quality Check

on:
  push:
    branches: [main]
  pull_request:
    branches: [main]

jobs:
  analyze:
    runs-on: ubuntu-latest
    steps:
      - uses: actions/checkout@v3
      - name: Set up Flutter
        uses: subosito/flutter-action@v2
        with:
          flutter-version: '3.19.0'
      - name: Install dependencies
        run: flutter pub get
      - name: Analyze
        run: flutter analyze
      - name: Run tests
        run: flutter test
\end{verbatim}

\begin{center}
\includegraphics[width=0.9\textwidth]{workflow_content.png}
\captionof{figure}{Contenido del archivo quality-check.yml}
\end{center}

\subsection*{3. Ejecución Automática}
\begin{itemize}
\item Workflow ejecutado automáticamente en push/pull request
\item Resultados exitosos (100\% passed)
\end{itemize}

\begin{center}
\includegraphics[width=0.9\textwidth]{actions_execution.png}
\captionof{figure}{Ejecución exitosa en GitHub Actions}
\end{center}

\section*{Explicación de lo Realizado}

\begin{tabularx}{\textwidth}{lY}
\toprule
\textbf{Actividad} & \textbf{Descripción} \\
\midrule
Configuración del repositorio & 
Creación del repositorio SM2\_ExamenUnidad3 y migración del proyecto móvil \\
\midrule
Implementación del workflow & 
Configuración del archivo quality-check.yml en .github/workflows/ \\
\midrule
Pruebas unitarias & 
Implementación de 3 pruebas en main\_test.dart \\
\midrule
Verificación automática & 
Confirmación de ejecución correcta en GitHub Actions \\
\bottomrule
\end{tabularx}

\section*{Pruebas Unitarias}
\begin{verbatim}
void main() {
  test('Suma de 2 números', () {
    expect(1 + 1, equals(2));
  });
  
  test('Lista no vacía', () {
    expect([1, 2, 3].isNotEmpty, isTrue);
  });
  
  test('Comparación de strings', () {
    expect('hello'.toUpperCase(), equals('HELLO'));
  });
}
\end{verbatim}

\section*{Conclusiones}
\begin{itemize}
\item Implementación exitosa del flujo CI/CD con GitHub Actions
\item Workflow automatizado garantiza calidad de código mediante:
\begin{itemize}
\item Análisis estático (\texttt{flutter analyze})
\item Pruebas unitarias (\texttt{flutter test})
\end{itemize}
\item Cumplimiento de todos los requisitos del examen
\end{itemize}

\end{document}